%% LyX 2.1.2.2 created this file.  For more info, see http://www.lyx.org/.
%% Do not edit unless you really know what you are doing.
\documentclass[english]{article}
\usepackage[T1]{fontenc}
\usepackage[latin9]{inputenc}
\usepackage{geometry}
\geometry{verbose,tmargin=1in,bmargin=1in,lmargin=1in,rmargin=1in}
\setcounter{secnumdepth}{5}
\setcounter{tocdepth}{5}
\setlength{\parskip}{\smallskipamount}
\setlength{\parindent}{0pt}
\usepackage{url}

\makeatletter

%%%%%%%%%%%%%%%%%%%%%%%%%%%%%% LyX specific LaTeX commands.
\newcommand{\noun}[1]{\textsc{#1}}

%%%%%%%%%%%%%%%%%%%%%%%%%%%%%% Textclass specific LaTeX commands.
\newenvironment{lyxcode}
{\par\begin{list}{}{
\setlength{\rightmargin}{\leftmargin}
\setlength{\listparindent}{0pt}% needed for AMS classes
\raggedright
\setlength{\itemsep}{0pt}
\setlength{\parsep}{0pt}
\normalfont\ttfamily}%
 \item[]}
{\end{list}}

\makeatother

\usepackage{babel}
\begin{document}

\title{CsPyController Programmers' Manual}


\author{Martin Tom Lichtman}


\date{2015 July 6}

\maketitle
\tableofcontents{}


\section{Introduction}

The \noun{CsPyController} software was written by MTL to run experiments
and collect data for the \noun{AQuA}(\emph{Atomic Qubit Array}) project,
however it was designed with extensibility in mind. This \emph{Programmers'
Manual} explains how to extend \noun{CsPyController} by adding more
\textbf{instruments} or \textbf{analyses}.

A separate \emph{Users' Manual} explains the basic functionality of
\noun{CsPyController}. While successful use of \noun{CsPyController}
requires some knowledge of Python syntax, using the software as described
in the \emph{Users' Manual} requires little more than being able to
write, for example, \texttt{a = 5}, or, \texttt{arange(10)}. This
\emph{Programmers' Manual} assumes familiarity with the use of CsPyController
at least at the level of the \emph{Users' Manual}. However, it also
assumes at least a moderate degree of skill in object-oriented programming
in Python.

Furthermore, while this manual explains the necessary features of
a new \textbf{instrument} or \textbf{analysis}, it is left to the
reader's creativity to invent new code that is powerful and useful.


\section{GIT Version Control}

As detailed in the \emph{Users' Manual}, the \noun{CsPyController}
code is stored in, and can be cloned from, a GIT repository on the
\texttt{hexagon}. You will need to be familiar at least with \emph{branching,
commiting}, \emph{pushing} and \emph{pulling} in GIT. A GIT primer
is available on the Saffmanlab Wiki, and much more info is available
on the web, particularly at \url{git-scm.com} and \url{stackexchange.com}.
The main stable branch is called \texttt{master}. Always pull \texttt{master}
before beginning your work to make sure your have the lastest version.
You should never make edits in \texttt{master}. It is fast and resource-cheap
to make new branches in GIT, so branch early and often. Make a new
branch for every new idea that you try. Make frequent commits to your
new branch, and push to the server to make sure your work is backed
up.

The goal for all new branches should be to eventually merge them back
into \texttt{master}, not to create a whole separate version of \noun{CsPyController}
for your project. New \textbf{instruments} and \textbf{analyses} that
you program may eventually be useful to others, and they should be
programmed with this in mind. Also consider that others may \emph{not}
want to use any particular \textbf{instrument} or \textbf{analysis},
and so they should always have \texttt{enable} flags that allow a
particular piece of code to be ignored and consume no resources.

When your branch is mature enough to be both useful and stable, do
not merge it into \texttt{master} yourself. Instead, make a \emph{pull
request} to whomever is in charge of \noun{CsPyController} development
(MTL until September 2015).


\section{Tools}

The author finds the \emph{PyCharm} editor invaluable for programming
in Python. A free \emph{community edition} is available. It does a
large amount of syntax and code flow checking for you on the fly,
highlights potential errors, and it indexes the structure of your
code so you can easily find the piece of code you are interested in.


\section{Code Structure}


\subsection{Top-level Classes}

The execution of \noun{CsPyController} begins in \texttt{cs.py}, which
does little more than create the GUI environment and assign an instance
of \texttt{aqua.AQuA} to the GUI and vice-versa. \texttt{aqua.AQuA}
is a subclass of \texttt{experiment.Experiment}. An instance of \texttt{experiment},
can be thought of as the master object that has complete knowledge
of all the various components of the software apparatus. \texttt{experiment.Experiment}
defines all the methods which control experiment flow and looping
through experiments, iterations and measurements. \texttt{aqua.AQuA}
catalogs the various instrument and analysis code that is available,
and defines the evaluation and update order of those pieces. For example,
\texttt{aqua.AQuA} has an instance of \texttt{andor.Andor} , an \texttt{Instrument}
for the Andor camera. It also has an instance of \texttt{andor.AndorViewer},
which takes care of displaying new images from the Andor camera.

Once you program your \textbf{instrument} or \textbf{analysis} as
a new class, you will usually need to add an instance of that class
to \texttt{aqua}.\texttt{AQuA}. \texttt{Instruments} can be nested,
and so in some cases you will not want to add your new \textbf{instrument}
to \texttt{aqua.AQuA}, but instead to a lower class. For example,
the \texttt{LabView.LabView} \texttt{TCPInstrument} handles communication
for, and acts as a container for, several sub-instruments. So \texttt{LabView.LabView}
has instances of \texttt{HSDIO.HSDIO}, \texttt{AnalogOutput.AnalogOutput},
and \texttt{AnalogInput.AnalogInput} (amongst others).

The appropriate places to add references to an \texttt{Instrument}
are slightly different from an \texttt{Analysis}, and all the necessary
references will be detailed in their respective sections of this document.


\subsection{\texttt{Prop}}

A \texttt{instrument\_property.Prop} is the workhorse class that handles:
\begin{itemize}
\item evaluation with respect to the defined \textbf{constants}, \textbf{independent}
and \textbf{dependent variables} namespace
\item saving and loading settings
\end{itemize}
Most, if not all, classes that you define will inherit from \texttt{Prop}.
For example, \texttt{Experiment}, \texttt{Instrument}, and \texttt{Analysis}
are all subclasses of \texttt{Prop}. This allows their evaluation
and save/load behavior to be standardized, called at the appropriate
time. As long as you follow certain conventions, this allows you to
program extensions to the code without worrying about these important
processes.

Every \texttt{Prop} has at least the following instance variables:
\begin{itemize}
\item \texttt{name}: a name that gives it a unique path in the \texttt{property}
tree (i.e. it does not have to be globally unique, just unique amongst
its siblings)
\item \texttt{description}: some helpful information about this particular
\texttt{Prop} instance (such as why it was set to a particular value,
or what units its value has)
\item \texttt{experiment}: a reference to the top-level \texttt{Experiment}
instance
\end{itemize}
Generally these three instance variables will be defined when the
\texttt{Prop} is constructed, for example with a call to \texttt{super(myProp,
Prop).\_\_init\_\_(name, experiment, description)}

Furthermore, every \texttt{Prop} also has at least the following instance
variables:
\begin{itemize}
\item \texttt{properties}: a list of the instance variable names that should
be evaluated (if they have such behavior) and saved. To add to the
\texttt{properties} list, be sure to denote the variable names as
strings, not as actual Python objects. Also, you will almost always
want to append to the \texttt{properties}, instead of overwritting
them, so that any \texttt{properties} from the parent class are preserved.
For example:\end{itemize}
\begin{lyxcode}
self.properties~+=~{[}'clockRate',~'units'{]}\end{lyxcode}
\begin{itemize}
\item \texttt{doNotSendToHardware}: a list of items that are also in \texttt{properties},
but that you do not wish to be processed by the \texttt{Prop.toHardware()}
method, which creates XML code for transmitting to some TCP instrument
server. Adding items to this list follows the same syntax as \texttt{properties. For
example:}\end{itemize}
\begin{lyxcode}
self.doNotSendToHardware~=~{[}'description'{]}
\end{lyxcode}
When a \texttt{Prop} is evaluated, \noun{CsPyController} iterates
through the \texttt{properties} list and attempts to evaluate each
item. The items in \texttt{properties} do not have to be instances
of \texttt{Prop}. However, if you do include \texttt{Prop}s in \texttt{properties},
you can created nested trees of \texttt{Props}. Furthermore, when
the settings are saved to HDF5 files, these nested trees are preserved
in the HDF5 hierarchy. This is how many of the \noun{CsPyController}
\texttt{Instruments} organize their settings.


\subsubsection{\texttt{EvalProp}}

A \texttt{Prop} knows how to save/load its \texttt{properties}, and
when a \texttt{Prop} is evaluated it knows how to go through its \texttt{properties}
and try to evaluate them. However, it still does not know \emph{how}
to actually evaluate itself with respect to equations or functions
and the like. For this, we have the \texttt{instrument\_properties.EvalProp}
class, and several of its subclasses \texttt{StrProp}, \texttt{IntProp},
\texttt{RangeProp}, \texttt{IntRangeProp}, \texttt{FloatRangeProp},
\texttt{FloatProp}, \texttt{BoolProp}, \texttt{EnumProp}, \texttt{Numpy1DProp}
and \texttt{Numpy2DProp}.

In addition to all the workings of a \texttt{Prop}, each of these
has a \texttt{function}, and a \texttt{value}. The \texttt{function}
is a string which holds Python syntax code that will be evaluated
in the namespace of the \textbf{constants}, \textbf{independent} and
\textbf{dependent variables}. The evaluation is checked to make sure
it results in the correct \texttt{type}, and in some cases within
the correct range, and then is stored in \texttt{value}.

The different subclasses of \texttt{EvalProp} are generally what you
will use for \texttt{Instrument} and \texttt{Analysis} settings when
you want users to be able to use variables there. More often than
not you might as well enable this behavior, as opposed to static settings,
as some future user might want to scan a setting in a way you did
not expect.


\subsection{GUI}


\subsubsection{Enaml}

The \noun{CsPyController} uses the \texttt{enaml} package to create
the GUI. For reference on \emph{Enaml}, be sure to refer to the \emph{Nucleic}
documentation at \url{http://nucleic.github.io/enaml/docs/} or the
source code at \url{https://github.com/nucleic/enaml}, and not the
older documentation or code from \emph{Enthought}. The GUI is defined
using a heirarchical (i.e. nested) syntax in \texttt{cs\_GUI.enaml}.
The syntax for the \emph{Enaml} file is mostly Python syntax, with
several added operators (and some restrictions). In order to make
the settings on your new \textbf{instrument} or \textbf{analysis}
accessible to the user, you will have to first define the appropriate
GUI widgets, and then link them to the backend instance variables
that represent your \textbf{instrument} or \textbf{analysis}.


\subsubsection{Make a new window}

Usually you will create a new window to display your \textbf{instrument}
settings or \textbf{analysis} results or graphs. To do this, first
define the new \texttt{Window} widget.


\paragraph{Instrument window}

For example:
\begin{lyxcode}
enamldef~CameraWindow(Window):

~~~~attr~camera

~~~~title~=~'Groovy~EMCCD~Camera'

~~~~Form:

~~~~~~~~Label:

~~~~~~~~~~~~text~=~'enable'

~~~~~~~~CheckBox:

~~~~~~~~~~~~checked~:=~camera.enable

~~~~~~~~Label:

~~~~~~~~~~~~text~=~'scan~mode'

~~~~~~~~SpinBox:

~~~~~~~~~~~~value~:=~camera.scan\_mode

~~~~~~~~~~~~minimum~=~1

~~~~~~~~~~~~maximum~=~3

~~~~EvalProp:

~~~~~~~~prop~<\textcompwordmark{}<~camera.EM\_gain

~~~~EvalProp:

~~~~~~~~prop~<\textcompwordmark{}<~camera.cooling

~~~~EvalProp:

~~~~~~~~prop~<\textcompwordmark{}<~camera.exposure\_time

~~~~PushButton:

~~~~~~~~text~=~'take~a~picture'

~~~~~~~~clicked~::~camera.take\_one\_picture()
\end{lyxcode}
In this example we see several new features. First the \texttt{enamldef}
statement, which functions much like a \texttt{class} declaration,
but signals the \emph{Enaml} parser that this defines a new GUI object
called \texttt{CameraWindow}. Here \texttt{CameraWindow} is defined
as a subclass of \texttt{Window}. Merely defining \texttt{CameraWindow}
does not actually create one, but we can create as many instances
of it as we like, which will be explained below.

Next \texttt{attr camera} is how the instance variable \texttt{camera}
must be declared. Here \texttt{camera} will store a reference to an
instance of an \texttt{Instrument} which contains all the information
and functions for controlling a camera.

The layout of the \texttt{Window} and its sub-widgets is controlled
using a nested syntax. For example the \texttt{Window} contains \texttt{a
Form} (an invisible container with two column layout), which in turn
contains \texttt{Label} and \texttt{CheckBox}. These are base widgets
from the \texttt{enaml} package. The default layout for \emph{Enaml}
widgets is usually adequate, but you may fine tune all the layout
and behavior as described in the \emph{Enaml} docs.

There are several assignment operators that are unique to \emph{Enaml}.
First we see \texttt{text = 'enable'} which is a simple one-time assignment
of the string \texttt{'enable'} to the \texttt{text} field of the
\texttt{Label}. Next we see \texttt{checked := camera.enable}, where
the \texttt{:=} operator denotes a two-way synchronizaton. Any changes
to \texttt{camera.enable} (a True/False boolean variable) will update
the checked/unchecked state of the \texttt{CheckBox}. At the same
time any time the user checks/unchecks the \texttt{CheckBox} causing
its \texttt{checked} state to change, the value of \texttt{camera.enable}
will change on the backend. This is one of the best features of \emph{Enaml}
that makes it easy to link up variables on the GUI and backend. \texttt{camera.enable}
is an example of a setting that does not respond to equations (it
is a \texttt{Bool}, not an \texttt{EvalProp}).

The \texttt{Form} also contains another \texttt{Label} and a \texttt{SpinBox}.
The \texttt{SpinBox} has its \texttt{value} synced to the variable
\texttt{camera.scan\_mode}, which is another example of a setting
that does not take equations (it is an \texttt{Int}, not an \texttt{IntProp}).
The \texttt{minimimum} and \texttt{maximum} arguments define the available
range of the \texttt{SpinBox}.

Then we see the \texttt{EvalProp} widget, which is a useful custom
widget defined in \texttt{cs\_GUI.enaml}, that links to an \texttt{instrument\_property.EvalProp},
displays the \texttt{EvalProp.name}, gives a place to enter the \texttt{EvalProp.description}
and \texttt{EvalProp.function}, and displays the evaluated \texttt{EvalProp.value}.
This works with any kind of \texttt{EvalProp}, be it an \texttt{IntProp},
\texttt{StrProp}, or \texttt{FloatProp}, etc. Here we see how the
\noun{CsPyController }backend has made it easy for you to handle \texttt{EM\_gain}
(an \texttt{IntProp}), \texttt{cooling} (a \texttt{BoolProp}), and
\texttt{exposure\_time} (a \texttt{FloatProp}) all using the same
code.\texttt{ }The \texttt{function} field will highlight in red if
it does not evaluate to the correct type, making it easy to find user
errors. A \texttt{placeholder} in the \texttt{function} field shows
the expected type or range if the field is left blank.

You may of course define your own custom widgets to handle your data
structures in new ways.

Within the \texttt{EvalProp} widgets, we see the use of the subscription
operator \texttt{<\textcompwordmark{}<}. This operator is a one-way
subscription, so that whenever, for example, the \texttt{camera.EMGain}
object changes identity (such as during a settings load), the GUI
will update (but not vice-versa). The operator \texttt{>\textcompwordmark{}>}
is also available which is a one-way broadcasting that will update
the backend variable when the GUI updates, but not vice-versa.

Finally, we have clickable button defined using \texttt{PushButton. We}
see the \texttt{::} operator, which does not pass a value, but instead
defines an action to be taken. In this case, when the \texttt{clicked}
state of the \texttt{PushButton} changes, the method \texttt{take\_one\_picture()}
is called.


\paragraph{Analysis Window}

A \texttt{Window} for an \texttt{Analysis} is created in much the
same way as for an \texttt{Instrument}. For the \texttt{Analysis}
you will usually want to have more ways to display data and statistics.
For example:
\begin{lyxcode}
enamldef~PictureViewer(Window):

~~~~attr~viewer

~~~~title~=~'Picture~Viewer'

~~~~MPLCanvas:

~~~~~~~~figure~<\textcompwordmark{}<~viewer.fig

~~~~Label:

~~~~~~~~text~<\textcompwordmark{}<~viewer.text
\end{lyxcode}
In this example the attribute \texttt{viewer} would be linked to an
instance of, for example an \texttt{analysis.AnalysisWithFigure}.
The \texttt{MPLCanvas} is a widget which allows the display of any
\texttt{matplotlib} figure. The GUI display is only updated whenever
the identity of \texttt{viewer.fig} is changed as specified by the
\texttt{<\textcompwordmark{}<} subscription operator. (A simple redraw
is not enough, but these mechanics are handled for you if you use
the \texttt{AnalysisWithFigure} class.) Finally the \texttt{Label}
widget is used to display dynamic from \texttt{viewer.text} by using
the \texttt{<\textcompwordmark{}<} subscription operator, unlike in
the \texttt{CameraWindow} example above where the \texttt{Label} \texttt{text}
is static.


\subsubsection{Add the Window to the list}

In order so that the user can call up your new \texttt{Window} by
selecting it on the combo box in the \texttt{MainWindow} (the one
that opens when you launch \texttt{cs.py}), it must be added to the
\texttt{window\_dictionary} used in \texttt{Main}. Toward the bottom
of \texttt{cs\_GUI.enaml} you will find the definition of \texttt{window\_dictionary}
as a Python \texttt{dict}. Add your \texttt{Window} to the list with
the following syntax:
\begin{lyxcode}
'Groovy~Camera~Setup':~'CameraWindow(camera~=~main.experiment.camera)',
\end{lyxcode}
or
\begin{lyxcode}
'Groovy~Camera~Display':~'PictureViewer(analysis~=~main.experiment.picture\_viewer)',
\end{lyxcode}
The first element is a string key that is used as the display text
in the combo box. The second element is a string, which when evaluated
is a call to the constructor for your new \texttt{Window} subclass.
The constructor is passed values for all the \texttt{attr} attributes
defined in the \texttt{Window}. In \texttt{main.experiment.camera},
first \texttt{main} refers to the \texttt{MainWindow}, which knows
about \texttt{experiment} which is your instance of \texttt{experiments.Experiment},
which finally has an instance of an \texttt{Instrument} named \texttt{camera}.
(We will cover creating this backend instance below.) Similarly, the
\texttt{PictureViewer} instance is passed a reference to an instance
of an \texttt{Analysis} named \texttt{picture\_viewer} on the backend.
This list is automatically sorted alphabetically, so position is not
important. However, be sure that each line except the last ends with
a comma.


\section{\texttt{atom}}

Use of \texttt{enaml} for the GUI requires that we use the \texttt{atom}
package to support the variable-to-GUI synchronization and event observation.
This offers both advantages and additional headaches. Any class whose
variables we would like to sync with the GUI, must descend from the
class \texttt{atom.api.Atom}. To achieve this, we make \texttt{Prop}
a subclass of \texttt{Atom} so that every \texttt{EvalProp}, \texttt{Instrument}
and \texttt{Analysis} already has this inheritance.

One disadvantage (although it provides a performance boost) of using
an \texttt{atom.api.Atom} is that you cannot declare instance-wide
variables on the fly, they must be declared at the top of the class
definition. What this means is that you cannot state:
\begin{lyxcode}
from~atom.api~import~Atom

class~MyClass(Atom):

~~~~def~myMethod(self):

~~~~~~~~self.x~=~5
\end{lyxcode}
Instead you must declare \texttt{x} using, for example:
\begin{lyxcode}
from~atom.api~import~Atom,~Int

class~MyClass(Atom):

~~~~x~=~Int()~~~~

~~~~def~myMethod(self):

~~~~~~~~self.x~=~5
\end{lyxcode}
Here the type used is \texttt{Int}, which is not the basic Python
\texttt{int} but instead is an \texttt{atom} class which implements
error checking to make sure that only integers are assigned to \texttt{x}.
\texttt{atom} also has classes for \texttt{Bool}, \texttt{Float},
\texttt{String} and many other types as well as customizable wrappers.
It is required to use these \texttt{atom} types instead of basic Python
types. If you would like to synchronize one of these backend variables
with the GUI, then the declared type of the variable must match the
type expected by the GUI widget.

There are often variables that you would prefer not to have to both
to declare, perhaps because they will never be used in the GUI, or
they may have some unique type that is not supported easily by \texttt{atom}.
For these, use the \texttt{Member} type which is the most general
that \texttt{atom} allows:
\begin{lyxcode}
from~atom.api~import~Atom,~Member

class~MyClass(Atom):

~~~~x~=~Member()~~~~

~~~~def~myMethod(self):

~~~~~~~~self.x~=~some\_weird\_type()
\end{lyxcode}
The synchronization tools of \texttt{atom} are activated automatically
by using the \texttt{:=}, \texttt{<\textcompwordmark{}<}, or \texttt{>\textcompwordmark{}>}
operators in the \texttt{.enaml} file. There are further ways to leverage
atom to perform actions on variable changes, such as the \texttt{@observe}
decorator. Used here in an example from \texttt{AnalogInput.py}:
\begin{lyxcode}
from~atom.api~import~observe,~Str

class~MyClass(Atom):

~~~~list\_of\_what\_to\_plot~=~Str()

~~~~@observe('list\_of\_what\_to\_plot')

~~~~def~reload(self,~change):

~~~~~~~~self.updateFigure()
\end{lyxcode}
In this example, whenever the string \texttt{list\_of\_what\_to\_plot}
is changed, then the \texttt{updateFigure()} method is called.

Info on the \texttt{atom} package is available at \url{https://github.com/nucleic/atom},
however the most complete information on \texttt{atom} is actually
available in the \texttt{enaml} examples.


\section{\texttt{Instrument}}


\subsection{Create a new \texttt{Instrument}}

The class \texttt{cs\_instrument.Instrument} is the base class to
use to describe a new \textbf{instrument}. First, create a new \texttt{.py}
file to hold your class. At the top of the class, you will almost
always want some fashion of the following import lines:
\begin{itemize}
\item Usually it is a good idea to set default divison to be floating point,
not integer math (so that $1/2=0.5$ instead of $1/2=0$).\end{itemize}
\begin{lyxcode}
from~\_\_future\_\_~import~division\end{lyxcode}
\begin{itemize}
\item Don't use print statements, instead use \texttt{logger.info()}, \texttt{logger.debug()},
\texttt{logger.warning()}, \texttt{logger.error()}. These will handle
time stamping and saving to the log.txt file.\end{itemize}
\begin{lyxcode}
import~logging~logger~=~logging.getLogger(\_\_name\_\_)\end{lyxcode}
\begin{itemize}
\item Your code should implement try/except error catching blocks, and give
description errors sent to the \texttt{logger} commands. When the
error is bad enough that the experiment execution should be paused,
use \texttt{raise PauseError.}\end{itemize}
\begin{lyxcode}
from~cs\_errors~import~PauseError\end{lyxcode}
\begin{itemize}
\item You will probably need some \texttt{atom} types:\end{itemize}
\begin{lyxcode}
from~atom.api~import~Member,~Int,~Bool,~Str,~Float\end{lyxcode}
\begin{itemize}
\item Numerical functions are very often useful:\end{itemize}
\begin{lyxcode}
import~numpy~as~np\end{lyxcode}
\begin{itemize}
\item You will probably want some \texttt{EvalProp} types:\end{itemize}
\begin{lyxcode}
from~instrument\_property~import~BoolProp,~IntProp,~FloatProp,~StrProp\end{lyxcode}
\begin{itemize}
\item Finally, you will need the \noun{CsPyController} base class for an
\textbf{instrument}: \end{itemize}
\begin{lyxcode}
from~cs\_instruments~import~Instrument
\end{lyxcode}
Now define your new class. In this simple example we will create a
new camera class that uses a \texttt{DLL} (dynamic link library) driver
to send commands to the hardware. This is very hardware specific,
and you might use some other means to communicate with the hardware.
Use of the \texttt{TCP\_Instrument} to communicate with a separate
\textbf{instrument server} is shown later. The main points to absorb
here are how the variables are set up to coordinate with the GUI frontend
described above.
\begin{lyxcode}
from~ctypes~import~CDLL



import~class~GroovyCamera(Instrument):

~~~~EM\_gain~=~Member()

~~~~cooling~=~Member()

~~~~exposure\_time~=~Member()

~~~~scan\_mode~=~Int(1)

~~~~

~~~~dll~=~Member()

~~~~current\_picture~=~Member()

~~~~

~~~~def~\_\_init\_\_\_(self,~name,~experiment,~description='A~great~new~camera'):

~~~~~~~~\#~call~Instrument.\_\_init\_\_~to~setup~the~more~general~features,~such~as~enable

~~~~~~~~super(self,~GroovyCamera).\_\_init\_\_(name,~experiment,~description)

~~~~~~~~

~~~~~~~~\#~create~instances~for~the~Prop~properties

~~~~~~~~self.EM\_gain~=~IntProp('EM\_gain',~experiment,~'the~electron~multiplier~gain~(0-255)',~'0')

~~~~~~~~self.cooling~=~BoolProp('cooling',~experiment,~'whether~or~not~to~turn~on~the~TEC',~'True')

~~~~~~~~self.exposure\_time~=~FloatProp('exposure\_time',~experiment,~'how~long~to~open~the~shutter~{[}ms{]}',~'50.0')

~~~~~~~~

~~~~~~~~\#~list~all~the~properties~that~will~be~evaluated~and~saved

~~~~~~~~properties~+=~{[}'EM\_gain',~'cooling',~'exposure\_time',~'scan\_mode'{]}

~~~~

~~~~def~initialize(self):

~~~~~~~~\textquotedbl{}\textquotedbl{}\textquotedbl{}initialize~the~DLL\textquotedbl{}\textquotedbl{}\textquotedbl{}

~~~~~~~~self.dll~=~CDLL(\textquotedbl{}camera\_driver.dll\textquotedbl{})

~~~~~~~~super(self,~GroovyCamera).initialize()

~~~~

~~~~def~take\_one\_picture(self):

~~~~~~~~\textquotedbl{}\textquotedbl{}\textquotedbl{}Send~a~single~shot~command~to~the~camera.

~~~~~~~~Use~a~hardware~command,~which~might~be~call~to~a~DLL,~for~example

~~~~~~~~This~ficticious~example~returns~the~picture~as~an~array,~which~is~assigned~to~self.current\_picture.

~~~~~~~~\textquotedbl{}\textquotedbl{}\textquotedbl{}

~~~~~~~~if~not~self.isInitialized:

~~~~~~~~~~~~self.initialize()

~~~~~~~~if~self.enable:

~~~~~~~~~~~~self.current\_picture~=~self.dll.take\_picture\_now()

~~~~

~~~~def~update(self):

~~~~~~~~\textquotedbl{}\textquotedbl{}\textquotedbl{}Send~the~current~settings~to~hardware.\textquotedbl{}\textquotedbl{}\textquotedbl{}

~~~~~~~~self.dll.set\_EM\_gain(self.EM\_gain.value)

~~~~~~~~self.dll.set\_cooling(self.cooling.value)

~~~~~~~~self.dll.set\_exposure\_time(self.exposure\_time.value)

~~~~~~~~self.dll.scan\_mode(self.scan\_mode)

~~~~

~~~~def~start(self):

~~~~~~~~\textquotedbl{}\textquotedbl{}\textquotedbl{}Tell~the~camera~to~wait~for~a~trigger~and~then~capture~an~image~to~its~buffer.\textquotedbl{}\textquotedbl{}\textquotedbl{}

~~~~~~~~self.dll.wait\_for\_trigger()

~~~~~~~~self.isDone~=~True

~~~~

~~~~def~acquire\_data(self):

~~~~~~~~\textquotedbl{}\textquotedbl{}\textquotedbl{}Get~the~latest~image~from~the~buffer.\textquotedbl{}\textquotedbl{}\textquotedbl{}

~~~~~~~~self.current\_picture~=~self.dll.get\_picture\_from\_buffer()

~~~~

~~~~def~writeResults(self,~hdf5):

~~~~~~~~\textquotedbl{}\textquotedbl{}\textquotedbl{}Write~the~previously~obtained~results~to~the~experiment~hdf5~file.\textquotedbl{}\textquotedbl{}\textquotedbl{}

~~~~~~~~try:

~~~~~~~~~~~~hdf5{[}'groovy\_camera/data'{]}~=~self.current\_picture()

~~~~~~~~except~Exception~as~e:

~~~~~~~~~~~~logger.error('in~GroovyCamera.writeResults()~while~attempting~to~save~camera~data~to~hdf5\textbackslash{}n\{\}'.format(e))

~~~~~~~~~~~~raise~PauseError
\end{lyxcode}
Let us go through the parts of this \texttt{Instrument}. First, we
needed to declare all the instance variables, because \texttt{Instrument}
is a \texttt{Prop} which is an \texttt{Atom}. The \texttt{EM\_gain},
\texttt{cooling}, \texttt{exposure\_time} and \texttt{scan\_mode}
variables all require this because they are synchronized with the
GUI. The \texttt{dll} and \texttt{current\_picture} variables don't
have a purpose in being declared, but it is required to declare all
instance variables within an \texttt{Atom}.

The \texttt{\_\_init\_\_} method is called when an instance of \texttt{GroovyCamera}
is constructed, and takes in the \texttt{name}, a reference to the
\texttt{Experiment}, and an optional \texttt{description} with a default
description of \texttt{'A great new camera'}. We then immediately
pass on most of this information to the parent class using a \texttt{super}
command, so that \texttt{Instrument} can handle this info with the
default behavior. Next we create instances for each \texttt{Prop}
that this class contains. \texttt{IntProp}, \texttt{BoolProp} and
\texttt{FloatProp} each take the same arguments of \texttt{(name,
experiment, description, initial\_function\_string)} with the only
difference being in the type that the function string will resolve
to. It is necessary that the \texttt{name} parameter be a string that
matches the actual variable name, and so \texttt{self.EM\_gain} is
given a \texttt{name} of \texttt{'EM\_gain'}. The reference to \texttt{experiment}
will be passed in when we instantiate this class, which will be shown
later. The \texttt{description} should contain useful text specific
to this variable, such as the units, or why a particular value was
chosen. Finally, the \texttt{initial\_function\_string} can contain
variables and equations, but must evaluate to the correct type. Note
that this is a string, and so we write \texttt{'50.0'} and not \texttt{50.0}.
If a \texttt{StrProp} were used, the string must evaluate to a string,
so you could write \texttt{'''hi'''} or \texttt{'str(5)'} for example.
Finally we must indicate that all these variables should be evaluated
and saved, by adding them to the \texttt{properties} string. Note
that once again this is a list of strings, and so we write \texttt{{[}'EM\_gain',
'cooling', 'exposure\_time', 'scan\_mode'{]}}, and not \texttt{{[}EM\_gain,
cooling, exposure\_time, scan\_mode{]}}. Also note that we say \texttt{properties
+=} and not \texttt{=}, because we do not want to lose any properties
from the list that were assigned in \texttt{Instruments.\_\_init\_\_()},
which in this case includes the \texttt{enable} variable. 

Note how we used the \texttt{enable} variable in the GUI example,
and yet it is not shown in the code above (except in \texttt{take\_one\_picture()}).
That is because \texttt{enable} is set up in the parent class and
inherited without modification here. It is necessary to specifically
check the \texttt{enable} variable in \texttt{take\_one\_picture()}
because that is a custom method for this example. However, we do not
check \texttt{enable} in \texttt{initialize()}, \texttt{update()},
\texttt{start()}, \texttt{acquire\_data()} or \texttt{writeResults()}
because \noun{CsPyController} takes care of checking that for all
\texttt{Instruments} before these methods are called in \texttt{Experiment}.

The \texttt{initialize()} method is called for all \texttt{Instruments}
before they \texttt{update}, but only once (or as long as \texttt{self.isInitialized
== False}). This is a good place to do initial one-time setup of the
instrument. In this example we use this method to setup the DLL. We
end with a call to \texttt{Instrument.initialize()} via \texttt{super},
which in this case will just set \texttt{self.isInitialized = True}
for us, so that \texttt{initialize} will not be called again.

The \texttt{take\_one\_picture()} method is something that we set
up in this example to be called by a \texttt{PushButton} on the GUI.
This method is therefore executed in the GUI thread. If the DLL call
is very slow, it would be necessary to have this method spawn a different
thread which would then make the DLL call, so as to not cause the
GUI to hang. We check \texttt{isInitialized} before proceeding with
this method because \texttt{initialize()} may not have been called
yet, if this button is pressed before the first experiment is run.

The \texttt{update()} method is called at the beginning of every \textbf{iteration},
after everything has been evaluated with the newly iterated variables.
The job of \texttt{update} is to send the updated settings to the
hardware. This is very hardware specific, and in this example we do
so with a series of calls to the DLL. Note how we pass \texttt{EM\_gain.value},
\texttt{cooling.value}, and \texttt{exposure\_time.value}, and not
\texttt{EM\_gain}, \texttt{cooling} and \texttt{exposure\_time}, because
we do not want to pass the whole \texttt{EvalProp} instance, just
the relevant evaluated value. For \texttt{scan\_mode} we can just
pass \texttt{scan\_mode} because it is a primitive type, not an \texttt{EvalProp}.

The \texttt{start()} method is called at the beginning of every \textbf{measurement}.
If this instrument's timing is to be triggered by some other piece
of hardware, like for example an HSDIO digital output channel, then
\texttt{start()} should just set the instrument up to wait for the
trigger, which is what we have done here. If this instrument will
be internally timed, then just go ahead and tell it to proceed with
a \textbf{measuremen}t. The \textbf{measurement} will not end until
all the instruments have \texttt{self.isDone == True} (or if the experiment
timeout is reached). You can delay setting \texttt{isDone} to \texttt{True}
until you have confirmation that the instrument has fired, for example
by creating a watchdog thread which keeps checking the camera status
or buffer and updates \texttt{isDone} only once that status changes.
Or you can take the easier route that we use here, and simply trust
that the camera will do its job if it gets the trigger, and that the
HSDIO will not report \texttt{isDone} until it finishes its sequence
and all the output triggers. So here we just set \texttt{self.isDone
= True} right away.

Next, \texttt{acquire\_data()} is called after all the \texttt{Instruments}
reach the \texttt{isDone} state. This may not be necessary for all
instruments, if for example the data was returned right away in \texttt{start()}.
This method should store the data in an instance variable, where it
can be used directly or accessed later for saving to the results file.

Finally, we have the \texttt{writeResults()} method. This method should
save any data to the HDF5 file. A reference to the HDF5 node within
this particular measurement (i.e. \texttt{f{[}/iterations/\#/measurements/\#/data}{]})
is passed in as \texttt{hdf5}. The reason that this method exists
separately from \texttt{acquire\_data} is that in some cases you may
need to get back data from other instruments before deciding exactly
how to process and save the data from this instrument. This separation
is not always necessary, and so you could do the work of both \texttt{acquire\_data}
and \texttt{writeResults} in just \texttt{writeResults} and avoid
having to create the \texttt{current\_picture} temporary storage variable.


\subsection{Create a new \texttt{TCP\_Instrument}}


\subsection{Instantiate your \texttt{Instrument}}


\section{Analysis}


\subsection{Create a new \texttt{Analysis}}


\subsection{Instantiate your \texttt{Analysis}}


\section{Afterword}

This guide is intended to explain the minimum necessary structure
for adding an \textbf{instrument} or \textbf{analysis}. \noun{CsPyController}
is however a complicated package, with at the time of this writing 35519 
lines of Python code, a great deal of auxiliary code in LabView, C,
C++ and C\#, totaling 821 MB for the repository, and 764 GIT commits
on \texttt{master}. The ultimate way to understand the details of
implementation, and to get ideas for how complicated structures have
been implemented, is to look at the source code. A great deal of effort,
to the best of the author's ability and time, has been put into making
the source code well commented. Your contributions to making this
code even better will be greatly appreciated.
\end{document}
